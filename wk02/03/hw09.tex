\documentclass[]{article}
\usepackage[margin={1in}]{geometry}
\usepackage[spanish]{babel}
\usepackage{amsmath}
\usepackage{listings}
\usepackage{verbatim}
\usepackage[obeyspaces]{url}

\lstset{breaklines=true} 
                       
\title{\textbf{Recolección de imágenes}}
\date{\small 16 / 01 / 21}
\author{\small Andrés Alam Sánchez Torres (A00824854)}

\begin{document}

\maketitle

\noindent    
Para esta actividad verifica que la instalación de Selenium se efectuara correctamente e invierte
tiempo revisando sus funcionalidades. En base a lo que aprendiste de la librería diseña una estrategia para
resolver la actividad integradora, escribe un pseudocódigo o un plan de acción de cómo resolverías este
problema.

\medspace

Los pasos par obtener las imágenes los describo a continuación:

    \begin{enumerate}
        \item Iniciar selenium en \path{http://image-net.org/}
        \item Enviar la palabra de clave al input de búsqueda, con name=''q'' y enviar la tecla enter. Lo que llevará
        a la página con los resultados.
        \item Obtener el código fuente de la tabla de resultados. Cada link tiene un elemento anchor con un link en 
        formato \path{synset?wnid=[wnid]}. 
        \item Obtener con regex los wnid de la tabla de resultados.
        \item Obtener una lista de todos los urls de imagenes con la api: \\
        \path{http://www.image-net.org/api/text/imagenet.synset.geturls?wnid=[wnid]}
        \item Después de este proceso se tendrá una lista de urls, la cual puede ser divida en dos 
        listas de 80\% y 20\% para los dos tipos de datos.
        \item Iterar cada lista, descargando cada imagen con opencv y guardando en el folder que le corresponde a 
        dicha lista.
        
    \end{enumerate}



\end{document}