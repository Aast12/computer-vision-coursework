\documentclass[]{article}
\usepackage[margin={1in}]{geometry}
\usepackage[spanish]{babel}
\usepackage{amsmath}
\usepackage{listings}
\usepackage{verbatim}
\usepackage[obeyspaces]{url}

\lstset{breaklines=true} 
                       
\title{\textbf{RLM}}
\date{\small 11 / 01 / 21}
\author{\small Andrés Alam Sánchez Torres (A00824854)}

\begin{document}

\maketitle

\begin{enumerate}
    \item Investiga el concepto de Homocedasticidad en el contexto de una RLM (adjunta la
    referencia de dónde obtuviste esta respuesta).

    Homocedasticidad se refiere a la caracteristica de un modelo de regresión lineal de mantener una 
    varianza constante en los errores. En otras palabras, el error de una predicción no varía mucho al 
    cambiar los valores de las variables independientes. Si sucede esta variación, entonces se debe 
    utilizar otro modelo no lineal que represente mejor el problema [1].


    \item Investiga los términos de correlación y covarianza, incluye cuáles son las fórmulas para
    calcularlas, menciona qué indican los valores de correlación y covarianza respecto a un
    set de datos. 

    Dados dos conjuntos de datos $X$ y $Y$ de $n$ elementos, con $x_{i} \in X, y_{i} \in Y$.

    La correlacion es un valor normalizado entre -1 y 1, que mide la relación lineal entre dos 
    variables, y se expresa de la forma:

    $$r_{XY} = \frac{\sum_{i=1}^{n}(x_{i} - \bar{x})(y_{i} - \bar{y})}{\sqrt{\sum_{i=1}^{n}(x_{i} - \bar{x})^2\sum_{i=1}^{n}(y_{i} - \bar{y})^2}}$$
    
    Si $r_{XY} > 0$, existe una correlación positiva, por lo que ambas variables varían en la misma 
    dirección, es decir, una tiende a incrementar cuando la otra también lo hace. Si $r_{XY} > 0$, 
    existe una correlación negativa, por lo que las variables varían en direcciones opuestas. 
    $r_{XY} = 0$ indica que las variables no estan correlacionadas. Asimismo, la magnitud $|r_{XY}|$
    indica la 'fuerza' con la que estan relacionadas las variables
    
    La covarianza es un valor para medir la relación que hay entre dos conjuntos de datos, respecto a 
    su variación. Se expresa de la forma:
    
    $$cov(X, Y) = \sum_{i=1}^{n}\frac{(x_{i} - \bar{x})(y_{i} - \bar{y})}{n}$$
    
    Si $cov(X, Y) > 0$, Y tiende a incrementar cuando X incrementa. Si $cov(X, Y) > 0$, Y tiende a 
    decrementar cuando X incrementa. $cov(X, Y) = 0$ indica que las variables no estan correlacionadas.

    Este valor es similar a la correlación; sin embargo, esta no es una medida normalizada, por lo que 
    su magnitud no refleja una misma relación para diferentes pares de variables[2].
    
    \item La siguiente tabla indica la relación entre las especificaciones de una tarjeta RAM y su
    precio, calcula la covarianza y la correlación.

    \begin{center}
        \begin{tabular}[]{| c | c |}
            \hline
            RAM (GB) (X) & Price (Y)\\
            \hline
            4 & 7,000 \\
            8 & 9,000 \\
            12 & 12,000 \\
            16 & 16,000 \\
            \hline
        \end{tabular}
    \end{center}

    $$\bar{x} = \frac{4 + 8 + 12 + 16}{4} = \frac{40}{4} = 10$$
    $$\bar{y} = \frac{7000, 9000, 12000, 16000}{4} = \frac{44000}{4} = 11000$$

    \begin{equation}
        \begin{split}
            \sum_{i=1}^{n}(x_{i} - \bar{x})(y_{i} - \bar{y}) =&\quad (4 - 10)(7000 - 11000) + (8 - 10)(9000 - 11000) \\ 
                        &\quad + (12 - 10)(12000 - 11000) + (16 - 10)(16000 - 11000)\\
                       =&\quad 60000\\
            \sum_{i=1}^{n}(x_{i} - \bar{x})^2 & = (4 - 10)^2 + (8 - 10)^2 + (12 - 10)^2 + (16 - 10)^2\\
                      =&\quad 80\\
            \sum_{i=1}^{n}(y_{i} - \bar{y})^2 & = (7000 - 11000)^2 + (9000 - 11000)^2 + (12000 - 11000)^2\\
                       &\quad + (16000 - 11000)^2\\
                      =&\quad 46000000\\
                      \\
            cov(X, Y) =&\quad \sum_{i=1}^{n}\frac{(x_{i} - \bar{x})(y_{i} - \bar{y})}{n}\\
                      =&\quad \frac{1}{n}\cdot\sum_{i=1}^{n}(x_{i} - \bar{x})(y_{i} - \bar{y})\\
                      =&\quad \frac{60000}{4}\\
                      =&\quad 15000\\
            r_{XY} =&\quad \frac{\sum_{i=1}^{n}(x_{i} - \bar{x})(y_{i} - \bar{y})}{\sqrt{\sum_{i=1}^{n}(x_{i} - \bar{x})^2\sum_{i=1}^{n}(y_{i} - \bar{y})^2}}\\
                   =&\quad \frac{60000}{\sqrt{80 \cdot 46000000}}\\
                   =&\quad 0.989071
        \end{split}
    \end{equation}


    \item Investiga el concepto de colinealidad y multicolinealidad en el contexto de una RLM,
    explica cómo se puede utilizar una matriz de correlación para identificar la
    multicolinealidad. (adjunta la referencia de dónde obtuviste esta respuesta). 
    
    La multicolinealidad se refiere al caso en el que múltiples variables independientes tienen 
    una gran correlacion entre ellas. Esto atenta con la idea de que las variables independientes 
    tienen que ser independientes entre sí. Puede generar problemas al tratar de crear un modelo 
    debido a los cambios que pueden ocurrir a una variable no afectan las demás variables.

    Para detectar esto se puede utilizar una matriz de correlación. Esta es una matriz cuyas columnas
    y filas estan compuestas de todas las variables independientes, y cada celda contiene el valor de 
    correlación entre las variables correspondientes a su fila y columna. De esta forma se puede 
    visualizar la correlación que hay entre una variable independiente y todas las demás. La multicolinealidad
    se puede identificar al encontrar valores en las celdas cercanos a 1 (excepto si es la correlación entre 
    una variable y si misma) [3].

\end{enumerate}

\newpage

\begin{thebibliography}{9}
    \bibitem{homoskedastic} 
    Corporate Finance Institute. (n. d.). Homoskedastic.\\
    \path{https://corporatefinanceinstitute.com/resources/knowledge/other/homoskedastic/}
    
    \bibitem{correlacion} 
    Vinuesa, P. (2020). Correlación: teoría y práctica.\\
    \path{https://www.ccg.unam.mx/~vinuesa/R4biosciences/docs/Tema8_correlacion.html#introduccion-el-concepto-de-correlacion}

    \bibitem{collinearity} 
    Wu, S. (2020). Multicollinearity in Regression.\\
    \path{https://towardsdatascience.com/multi-collinearity-in-regression-fe7a2c1467ea}

    \end{thebibliography}

\end{document}