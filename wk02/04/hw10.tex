\documentclass[]{article}
\usepackage[margin={1in}]{geometry}
\usepackage[spanish]{babel}
\usepackage{amsmath}
\usepackage[obeyspaces]{url}

\title{\textbf{Procesamiento de imágenes}}
\date{\small 14 / 01 / 21}
\author{\small Andrés Alam Sánchez Torres (A00824854)}

\begin{document}

\maketitle

\begin{enumerate}
    \item Determina la matriz de salida al convolucionar la siguiente matriz de entrada:
    
    \begin{center}
        \begin{tabular}[]{| c | c | c | c | c | }
            \hline
            3 & 5 & 2 & 8 & 1 \\
            \hline
            9 & 7 & 5 & 4 & 3 \\
            \hline
            2 & 0 & 6 & 1 & 6 \\
            \hline
            6 & 3 & 7 & 9 & 2 \\
            \hline
        \end{tabular}
    \end{center}

    Con el siguiente Kernel:

    \begin{center}
        \begin{tabular}[]{| c | c | c |}
            \hline
            1 & 0 & 0 \\
            \hline
            1 & 1 & 0 \\
            \hline
            0 & 0 & 1 \\
            \hline
        \end{tabular}
    \end{center}

    \begin{center}
        \begin{tabular}[]{| c | c | c | c | c | }
            \hline
            3 & 5 & 2 & 8 & 1 \\
            \hline
            9 & 7 & 5 & 4 & 3 \\
            \hline
            2 & 0 & 6 & 1 & 6 \\
            \hline
            6 & 3 & 7 & 9 & 2 \\
            \hline
        \end{tabular}
        *
        \begin{tabular}[]{| c | c | c |}
            \hline
            1 & 0 & 0 \\
            \hline
            1 & 1 & 0 \\
            \hline
            0 & 0 & 1 \\
            \hline
        \end{tabular}
        =
        \begin{tabular}[]{| c | c | c | }
            \hline
            25 & 18 & 17 \\
            \hline
            18 & 22 & 17 \\
            \hline
        \end{tabular}
    \end{center}


    \item  Cuando realizamos una convolución con una matriz de entrada $A_{6x6}$ y un kernel $C_{3x3}$,
    obtuvimos una matriz de salida con dimensiones $D_{4x4}$. Dada una matriz de entrada $A_{NxN}$ y
    un kernel $C_{FxF}$, ¿cuáles serían las dimensiones de nuestra matriz de salida $D$?.

    (N - F + 1) x (N - F + 1)



\end{enumerate}



\end{document}